\documentclass{aa}
% \documentclass[referee]{aa}
\usepackage[varg]{txfonts}
\usepackage[separate-uncertainty=true]{siunitx}
\usepackage[version=3]{mhchem}

\sisetup{range-units = brackets}

\def\eps{\varepsilon}
\def\aap{A\&A}
\def\eprint{e-prints}
\def\apj{ApJ}
\def\apjs{ApJS}
\def\apjl{ApJL}
\def\mnras{MNRAS}
\def\aj{AJ}
\def\nat{Nature}
\def\aaps{A\&A Supp.}
\def\prd{Phys. Rev. D}
\def\prl{Phys. Rev. Lett.}
\def\araa{ARA\&A}       % Annual Review of Astron and Astrophys

\begin{document}


\title{SWEET-Cat update and MOOGme}
\subtitle{A new minimization procedure for high quality spectra}


\author{ D.~T.~Andreasen\inst{1,2}
    \and S.~G.~Sousa\inst{1}
    \and N.~C.~Santos\inst{1,2}
    \and M.~Tsantaki\inst{3}
    \and G.~Teixeira\inst{1}
    \and L.~Andrez\inst{4}
    \and A.~Mortier\inst{5}
}


\institute{
Instituto de Astrof\'isica e Ci\^encias do Espa\c{c}o, Universidade do Porto, CAUP, Rua das Estrelas, 4150-762 Porto, Portugal
\email{daniel.andreasen@astro.up.pt}
\and
Departamento de F\'isica e Astronomia, Faculdade de Ci\^encias, Universidade do Porto, Rua Campo Alegre, 4169-007 Porto, Portugal
\and
Astronomy, Mexico
\and
IAC
\and
Astronomy, Scotland
}





\date{Received ...; accepted ...}

\abstract
% Context
{}
% Aims
{}
% Methods
{}
% Results
{}
% Conclusions
{}



\keywords{data reduction: high resolution spectra --
          stars individual: Arcturus --
          stars individual: HD010853}
\maketitle



\section{Introduction}
\label{sec:introduction}
The study of extrasolar planetary systems is an established field of research.
To date, over 3200 extrasolar planets have been discovered around solar-type
stars\footnote{For an updated table we refer to \url{http://ww.exoplanet.eu}}.
Most of these have been found thanks to the incredible precision achieved in
photometric transit and radial velocity. Especially the latest announcement
from the \emph{Kepler} space mission with 1284 confirmed exoplanets
\citep{Morton2016}. The increasing number of exoplanets allow us to do
statistical studies of the newfound worlds by analyzing their internal
structure, atmospheric composition, with more.

A key aspect to this progress is the characterization of the planet host stars.
For instance, precise and accurate stellar radii are critical if we want to
measure oresice values of the radius of a transiting planet
\citep[see e.g.][]{Torres2012}. The determination of the stellar radius
is in turn dependent on the quality of the derived stellar parameters such as
the effective temperature.



\section{MOOGme}
\label{sec:MOOGme}

MOOGme (acronym for MOOG made easy) is a new tool for analyzing spectra.
MOOGme is written in Python and works as a wrapper around MOOG
\citep{Sneden1973}, and ARES \citep{Sousa2015a} for an all-in-one tool.
MOOG is a radiative transfer code under the assumption of local
thermodynamic equilibrium (LTE). And ARES is a tool to measure equivalent
widths (EW) automatically from a spectrum given a line list. MOOGme has
four different functions: Measure EWs with ARES, synthetic fitting, EW method,
and abundances, all described below.

\subsection{EW measurements}
\label{sub:EW_measurements}
EW measurements are important for the EW method and to obtain abundances. This
can be done manually using a tool like IRAF, but often when dealing with a large
sample of stars this is not a suitable way to deal with the problem. Therefore
tools like ARES exists which can measure the EW of spectral lines automatically.
To use this mode of MOOGme, ARES has to be installed and be in the PATH. Then
MOOGme just need a spectrum (format should be 1D for ARES to read it) and a line
list. For the latter, MOOGme is shipped with some line lists ready to use, in
the format suitable for MOOGme. The output will be a line list in the format
required for MOOG. The output can be used for either the EW method or the
abundance method, both described below.


\subsection{Synthetic fitting}
\label{sub:Synthetic_fitting}

FOR MARIA :)



\subsection{EW method}
\label{sub:EW_method}
This is the second standard method for obtaining parameters from stellar
spectra. Here measured EWs are used to calculate abundances using a given
stellar atmosphere model with a given set of atmospheric parameters,
effective temperature ($T_\mathrm{eff}$), surface gravity ($\log g$),
metallicity ([Fe/H], where iron often is used as a proxy), and the micro
turbulence (\xi_\mathrm{micro}). By removing correlations between the measured
abundances (through the measured EWs) and the excitation potential and reduced
EW ($\log(EW/\lambda)$) we can constrain $T_\mathrm{eff}$ and $\xi_\mathrm{micro}$. By
obtaining ionization balance between \ion{Fe}{I} and \ion{Fe}{II}, that is
the average abundance of all \ion{Fe}{I} lines are equal to the average
abundance of all \ion{Fe}{II} lines, we constrain $\log g$. Last, we change
the input [\ion{Fe}/\ion{H}] to match that of the average output
[\ion{Fe}/\ion{H}]. Hence we have four criteria to minimize simultaneously:

\begin{enumerate}
    \item The slope between abundance and excitation potential ($a_\mathrm{EP}$).
    \item The slope between abundance and reduced EW ($a_\mathrm{RW}$).
    \item The difference between the average abundances of \ion{Fe}{I} and
          \ion{Fe}{II} ($\Delta$\ion{Fe}).
    \item Input and output metallicity.
\end{enumerate}

There exists many minimization routines available in Python. Most commonly
known are the ones from the SciPy ecosystem\footnote{\url{http.scipy.org}}.
There are some pros and cons with using proprietary minimization routines.
Pros are that it is already written, and usually there are good documentation
in libraries such as SciPy. Cons in this situation is, that most minimization
routines are not able to handle multiple criteria at once. A work around is
to combine the criteria into one single criteria by e.g. adding them
quadratically and minimize that expression instead. The minimization routines
are also not physical in the sense that they are not optimized for the problem.
These two cons was incitement for writing a minimization optimized for our
problem. Here is how it works.

\begin{enumerate}
    \item Run MOOG once with a user defined initial parameters (default is
          solar) and calculate $a_\mathrm{EP}$, $a_\mathrm{RW}$, and
          $\Delta$\ion{Fe}.
    \item Change the atmospheric parameters ($T_\mathrm{eff}$, $\log g$,
          [\ion{Fe}/\ion{H}], $\xi_\mathrm{micro}$) according to the size of the
          indicator. A parameter is only changed if it is not fixed.
    \begin{itemize}
        \item $a_\mathrm{EP}$: Indicator for $T_\mathrm{eff}$. If this value
              is positive, then increase $T_\mathrm{eff}$.
        \item $a_\mathrm{RW}$: Same as above but for $\xi_\mathrm{micro}$.
        \item $\Delta$\ion{Fe}: Same as above but for $\log g$. Positive
              $\Delta$\ion{Fe} means $\log g$ should be decreased.
    \end{itemize}
    \item For [\ion{Fe}/\ion{H}] it is changed to the output [\ion{Fe}/\ion{H}]
          in each iteration (if free).
    \item If the new parameters have already been used in a previous, then
          change them slightly. This is done by drawing a random number from
          a Gaussian distribution with a mean at the previous value and a sigma
          equal to the absolute value of the indicator.
    \item Calculate a new atmospheric model by interpolating a grid so we have
          the requested parameters and run MOOG once again.
    \item For each iteration save the parameters used and the quadratic sum of
          the indicators. If we do not reach convergence, then return the best
          found parameters.
\end{enumerate}

By using the indicators like this, we can relative fast reach convergence.
There are some interdependencies among the indicators. E.g. by changing
$T_\mathrm{eff}$ all indicators will be affected, however the effect is
strongest for $a_\mathrm{EP}$.

\subsubsection{Options}
\label{subs:EWoptions}
It is possible to run the EW method with a set of different options which
will be described here.

\begin{itemize}
    \item[model] Change which type of model atmosphere to use. Currently the
          following are available: Kurucz '95, Marcs, and Kurucz from APOGEE.
    \item[weights] Calculate the slopes with an weight based on the distance
          from the mean.
    \item[fix_teff] Fix $T_\mathrm{eff}$. Same is available for $\log g$
          (fix_logg), [\ion{Fe}/\ion{H}] (fix_feh), and $\xi_\mathrm{micro}$
          (fix_vt).
    \item[iterations] The maximum number of iterations (160 by default).
    \item[EPcrit] The criteria for $a_\mathrm{EP}$ to reach convergence (0.001
          by default).
    \item[RWcrit] The criteria for $a_\mathrm{RW}$ to reach convergence (0.003
          by default).
    \item[ABdiffcrit] The criteria for $\Delta$\ion{Fe} to reach convergence
          (0.01 by default).
    \item[MOOGv] The version of MOOG being used (2014 by default).
    \item[outlier] Remove outliers after the first run with the minimization
          routine and restarting the minimization from the previous best
          parameters. The options are to remove all outliers above $3\sigma$
          once or iteratively, or remove one outlier above $3\sigma$ once or
          iteratively.
    \item[autofixvt] If the minimization routine does not converge and
          $\xi_\mathrm{micro}$ is close to 0 or 10 with a significant
          $a_\mathrm{RW}$, then fix $\xi_\mathrm{micro}$.
    \item[teffrange] This is a special option for two of the line list shipped
          with MOOGme. The ones from \citet{Sousa2015b} and \citet{Tsantaki2013}.
          The latter is a subset of the former optimized for cooler stars.
          If this option is set, and the former line list is used for a cool
          star ($T_\mathrm{eff}$ found after the minimization), then remove
          the lines from the used line list to match the latter line list and
          rerun the minimization routine.
    \item[refine] After the minimization is done, run it again from the best
          found parameters but with more strict criteria. If this option is set,
          it will always be the last step (after removal of outliers and the
          use of teffrange).
\end{itemize}

If $\xi_\mathrm{micro}$ is fixed it is changed at each iteration according to
an empirical relation. For dwarfs it follows the one presented in
\citet{Tsantaki2013} and for giants it follows the one presented in
\citet{Adibekyan2015} (THIS IS NOT TRUE. NOT RIGHT REFERENCE FOR VARDAN).



\subsection{Abundance method}
\label{sub:Abundance_method}

We made a mode to calculate abundances for different elements based on the
measured EW. Here we require a line list with the EW of the elements and
the corresponding atmospheric parameters for the star of interest. We provide
a line list with XXX elements ready to use. The results are saved to a table.











\section{New spectroscopic parameters for 65 planet hosts}
\label{sec:results}
Here we present the sample of 66 stars. We were unable to derive parameters for
HD77065. This is a spectroscopic binary according to \cite{Pourbaix2004}.

The remaining 65 stars are presented in Table~\ref{tab:results}


\begin{table*}[htb!]
    \caption{The derived parameters for the 65 stars in our sample.}
    \label{tab:results}
    \centering
    \begin{tabular}{llllll}
      \hline\hline
        Star      & $T_\mathrm{eff}$ (K) &  $\log g$ (dex)     &  [Fe/H] (dex)        &  $\xi_\mathrm{micro}$ (km/s) & $\xi_\mathrm{micro}$ fixed? \\
      \hline
      WASP-762    &    $6347 \pm 52 $    &  $4.29 \pm 0.08$    &  $ 0.36 \pm 0.04$    &  $1.73 \pm 0.06$             &   no  \\
      WASP-822    &    $6563 \pm 55 $    &  $4.29 \pm 0.10$    &  $ 0.18 \pm 0.04$    &  $1.93 \pm 0.08$             &   no  \\
      WASP-882    &    $6450 \pm 61 $    &  $4.24 \pm 0.06$    &  $ 0.03 \pm 0.04$    &  $1.79 \pm 0.09$             &   no  \\
      WASP-952    &    $5799 \pm 31 $    &  $4.29 \pm 0.05$    &  $ 0.22 \pm 0.03$    &  $1.18 \pm 0.04$             &   no  \\
      WASP-972    &    $5723 \pm 52 $    &  $4.37 \pm 0.07$    &  $ 0.31 \pm 0.04$    &  $1.03 \pm 0.08$             &   no  \\
      WASP-992    &    $6324 \pm 89 $    &  $4.70 \pm 0.11$    &  $ 0.27 \pm 0.06$    &  $1.83 \pm 0.12$             &   no  \\
      HATS-12     &    $5969 \pm 46 $    &  $4.61 \pm 0.06$    &  $-0.04 \pm 0.04$    &  $1.06 \pm 0.08$             &   no  \\
      Qatar-22    &    $4637 \pm 316$    &  $4.23 \pm 0.61$    &  $ 0.09 \pm 0.17$    &  $0.63 \pm 0.83$             &   no  \\
      WASP-442    &    $5612 \pm 80 $    &  $4.47 \pm 0.30$    &  $ 0.17 \pm 0.06$    &  $1.32 \pm 0.13$             &   no  \\
      HAT-P-462   &    $6421 \pm 121$    &  $4.53 \pm 0.14$    &  $ 0.16 \pm 0.09$    &  $1.67 \pm 0.18$             &   no  \\
      WASP-522    &    $5197 \pm 83 $    &  $4.47 \pm 0.30$    &  $ 0.15 \pm 0.05$    &  $1.16 \pm 0.14$             &   no  \\
      WASP-722    &    $6570 \pm 85 $    &  $4.71 \pm 0.13$    &  $ 0.15 \pm 0.06$    &  $2.30 \pm 0.15$             &   no  \\
      WASP-752    &    $6203 \pm 46 $    &  $4.42 \pm 0.22$    &  $ 0.24 \pm 0.03$    &  $1.45 \pm 0.06$             &   no  \\
      HAT-P-422   &    $5903 \pm 66 $    &  $4.29 \pm 0.10$    &  $ 0.34 \pm 0.05$    &  $1.19 \pm 0.08$             &   no  \\
      HATS-52     &    $5383 \pm 91 $    &  $4.40 \pm 0.22$    &  $ 0.08 \pm 0.06$    &  $0.91 \pm 0.14$             &   no  \\
      HD2855072   &    $4620 \pm 126$    &  $4.42 \pm 0.61$    &  $ 0.04 \pm 0.06$    &  $0.74 \pm 0.43$             &   no  \\
      HR2282      &    $5042 \pm 42 $    &  $3.30 \pm 0.09$    &  $ 0.07 \pm 0.03$    &  $1.14 \pm 0.04$             &   no  \\
      SAND3642    &    $4457 \pm 104$    &  $2.26 \pm 0.20$    &  $-0.04 \pm 0.06$    &  $1.60 \pm 0.11$             &   no  \\
      Aldebaran   &    $5279 \pm 223$    &  $4.53 \pm 0.40$    &  $ 0.14 \pm 0.12$    &  $3.05 \pm 0.41$             &   no  \\
      Gl785       &    $5087 \pm 48 $    &  $4.30 \pm 0.10$    &  $-0.01 \pm 0.03$    &  $0.69 \pm 0.10$             &   no  \\
      HD120084    &    $4969 \pm 40 $    &  $2.94 \pm 0.14$    &  $ 0.12 \pm 0.03$    &  $1.41 \pm 0.04$             &   no  \\
      HD192263    &    $4946 \pm 46 $    &  $4.43 \pm 0.14$    &  $-0.05 \pm 0.02$    &  $0.66 \pm 0.12$             &   no  \\
      HD207229    &    $4957 \pm 49 $    &  $2.83 \pm 0.09$    &  $ 0.04 \pm 0.04$    &  $1.49 \pm 0.05$             &   no  \\
      HD219134    &    $4767 \pm 70 $    &  $4.32 \pm 0.17$    &  $-0.00 \pm 0.04$    &  $0.59 \pm 0.24$             &   no  \\
      HD81688     &    $4870 \pm 30 $    &  $2.50 \pm 0.14$    &  $-0.26 \pm 0.03$    &  $1.50 \pm 0.03$             &   no  \\
      HD82886     &    $5124 \pm 22 $    &  $3.30 \pm 0.05$    &  $-0.25 \pm 0.02$    &  $1.15 \pm 0.03$             &   no  \\
      HD85503     &    $4605 \pm 94 $    &  $2.61 \pm 0.26$    &  $ 0.25 \pm 0.06$    &  $1.64 \pm 0.11$             &   no  \\
      HD87883     &    $4917 \pm 68 $    &  $4.34 \pm 0.19$    &  $ 0.02 \pm 0.03$    &  $0.46 \pm 0.21$             &   no  \\
      HIP11915    &    $5770 \pm 14 $    &  $4.47 \pm 0.03$    &  $-0.06 \pm 0.01$    &  $0.95 \pm 0.02$             &   no  \\
      omiUma      &    $5499 \pm 52 $    &  $3.36 \pm 0.07$    &  $-0.01 \pm 0.05$    &  $1.98 \pm 0.06$             &   no  \\
      11Com       &    $4911 \pm 38 $    &  $2.68 \pm 0.08$    &  $-0.20 \pm 0.03$    &  $1.56 \pm 0.04$             &   no  \\
      HD102272    &    $5351 \pm 135$    &  $3.92 \pm 0.33$    &  $-0.34 \pm 0.11$    &  $1.16 \pm 0.20$             &   no  \\
      HD104985    &    $4809 \pm 48 $    &  $2.73 \pm 0.08$    &  $-0.26 \pm 0.04$    &  $1.65 \pm 0.05$             &   no  \\
      HD114762    &    $6058 \pm 83 $    &  $4.71 \pm 0.09$    &  $-0.78 \pm 0.05$    &  $0.00 \pm 0.23$             &   no  \\
      HD114762    &    $6061 \pm 83 $    &  $4.70 \pm 0.08$    &  $-0.78 \pm 0.05$    &  $0.02 \pm 0.26$             &   no  \\
      HD136512    &    $4915 \pm 33 $    &  $2.74 \pm 0.08$    &  $-0.14 \pm 0.03$    &  $1.57 \pm 0.04$             &   no  \\
      HD152581    &    $5355 \pm 82 $    &  $3.65 \pm 0.18$    &  $-0.39 \pm 0.07$    &  $0.60 \pm 0.15$             &   no  \\
      HD155358    &    $5917 \pm 51 $    &  $4.12 \pm 0.08$    &  $-0.55 \pm 0.04$    &  $1.06 \pm 0.08$             &   no  \\
      HD170693    &    $4547 \pm 55 $    &  $2.23 \pm 0.10$    &  $-0.31 \pm 0.03$    &  $1.54 \pm 0.05$             &   no  \\
      HD220842    &    $6027 \pm 30 $    &  $4.35 \pm 0.05$    &  $-0.08 \pm 0.03$    &  $1.19 \pm 0.04$             &   no  \\
      HD221345    &    $4797 \pm 44 $    &  $2.58 \pm 0.11$    &  $-0.23 \pm 0.03$    &  $1.58 \pm 0.04$             &   no  \\
      HD233604    &    $4925 \pm 44 $    &  $2.79 \pm 0.11$    &  $-0.15 \pm 0.03$    &  $1.62 \pm 0.05$             &   no  \\
      HD37124     &    $5468 \pm 32 $    &  $4.28 \pm 0.04$    &  $-0.43 \pm 0.03$    &  $0.67 \pm 0.07$             &   no  \\
      HD81688a    &    $4906 \pm 29 $    &  $2.69 \pm 0.06$    &  $-0.21 \pm 0.02$    &  $1.60 \pm 0.03$             &   no  \\
      HD82886     &    $5252 \pm 66 $    &  $3.67 \pm 0.13$    &  $-0.41 \pm 0.06$    &  $0.06 \pm 0.10$             &   no  \\
      HD97658     &    $5182 \pm 43 $    &  $4.50 \pm 0.12$    &  $-0.29 \pm 0.03$    &  $0.77 \pm 0.11$             &   no  \\
      Kepler-444  &    $5163 \pm 40 $    &  $4.41 \pm 0.11$    &  $-0.50 \pm 0.03$    &  $0.78 \pm 0.10$             &   no  \\
      WASP-100    &    $6853 \pm 209$    &  $4.15 \pm 0.26$    &  $-0.30 \pm 0.12$    &  $1.87 \pm 0.02$             &  yes  \\
      HAT-P-242   &    $6470 \pm 181$    &  $4.75 \pm 0.26$    &  $-0.41 \pm 0.10$    &  $1.40 \pm 0.03$             &  yes  \\
      HAT-P-392   &    $6745 \pm 236$    &  $4.91 \pm 0.46$    &  $-0.21 \pm 0.12$    &  $1.53 \pm 0.04$             &  yes  \\
      WASP-612    &    $6265 \pm 168$    &  $4.21 \pm 0.21$    &  $-0.38 \pm 0.11$    &  $1.44 \pm 0.02$             &  yes  \\
      HD70573     &    $5889 \pm 186$    &  $4.32 \pm 0.27$    &  $-0.42 \pm 0.13$    &  $1.14 \pm 0.01$             &  yes  \\
      \hline
    \end{tabular}
\end{table*}





\section{Conclusion}
\label{sec:conclusion}




\begin{acknowledgements}

This work was supported by Funda\c{c}\~ao para a Ci\^encia e a
Tecnologia (FCT) through the research grants UID/FIS/04434/2013 and
PTDC/FIS-AST/1526/2014. N.C.S., and S.G.S. acknowledge the support from
FCT through Investigador FCT contracts of reference IF/00169/2012, and
IF/00028/2014, respectively, and POPH/FSE (EC) by FEDER funding through
the program “Programa Operacional de Factores de Competitividade
- COMPETE”. E.D.M. and B.J.A. acknowledge the support from FCT in
form of the fellowship SFRH/BPD/76606/2011 and SFRH/BPD/87776/2012,
respectively. This work also benefit from the collaboration of a
cooperation project FCT/CAPES - 2014/2015 (FCT Proc 4.4.1.00 CAPES).

This research has made use of the SIMBAD database operated at CDS,
Strasbourg (France).

\end{acknowledgements}


\bibpunct{(}{)}{;}{a}{}{,}
\bibliographystyle{aa}
\bibliography{thesis}


\begin{appendix}
\section{An appendix}


\end{appendix}




\end{document}
